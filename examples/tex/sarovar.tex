\documentclass{article}
\usepackage{river_valley}
\begin{document}

\section{Sarovar.org: \textsc{floss} Project Hosting Facility}
Sarovar.org is a \textsc{floss} project hosting facility similar to
SourceForge.net. Sarovar.org is the first of its kind in India. This
document enlists the facilities available at Sarovar.org.

\subsection{Project Home Page}
While registering a project at Sarovar.org, along with other
facilities, the project is given with a home page of its own which can
be managed by project admins with out any help from Sarovar.org crew.

\subsection{Revision/Version Control System}

It is required for a software project to track changes in its source
code. A system which provides this facility is called a revision
control system. Which allows to keep track of the changes in source
code, see differences of a given file with one of its older versions,
check out an older version and to revert any changes made. An
efficient revision control system achieves this with out keeping
multiple copies of the source code and it is achieved by keeping
differences of successive versions of files. \ref{1}

It is very common that multiple version of a software (generally a
collection of source code files with each file has its own revision
history) being maintained concurrently. A version control system is
used in this case.

Sarovar.org provides the most widely used, an Opensource system named
{\em \textsc{cvs}} for this functionality. Developers of the project
can checkout and commit changes to the {\em source tree} and it can be
checked out by anyone who wants to see the source code through {\em
  anonymous login}. Sarovar.org also provides a Web interface to the
\textsc{cvs} system which allows those who don't have a \textsc{cvs}
client to view the source code.

\subsection{File Release and Download Facility}

Though \textsc{cvs} keeps track of the changes in source code, a
non-developer interested in a software package being hosted can
download a pre-packed release from Sarovar.org. A developer can check
out and {\em tag} present versions of source files at \textsc{cvs} and
pack it as a release with a version number. Sarovar.org also provides
a facility to add a release notes which helps the uses to see what are
the changes in that release. Previous versions are also kept in case
someone wants to get an older version of the package.

It will be interesting to keep track of the download and definitely it
would be an encouragement for the authors; a high download count means
a high degree of usefulness for that package. Sarovar.org keeps a
download counter for each project running in it.

\subsection{Bug Tracking System}

This is an excellent tool to report, monitor, and manage bugs. Anyone,
even without logging into Sarovar.org can report a bug. The manager of
the project can assign bug fixing to one or more of the developers. It
displays the status (Open/Closed/Deleted) of the reported bugs.

\subsection{Patch Tracking}

A lot of users generously contribute modifications back to the author
for inclusion into the main {\em source tree}. Sarovar.org provides a
mechanism for this too.

\subsection{Support Request/Feature Request Tracking}

Users can submit support/feature request to the authors/maintainers of
the project.

\subsection{Public Forums}
Public forums helps to hold open discussions on topics pertaining to
the project being hosted. Sarovar.org automatically set up two forums
(one for general discussion and the other for getting public help)
upon registering a project. More forums can be setup as and when
required.

\subsection{Documentation Management}

Any work without documentation is virtually unusable. So, writing
documentation is as important as writing the software package.
Sarovar.org provides a documentation management facility which allows
submitting and updating new documentations.

\subsection{Mailing Lists}

Mail list is important to the success of any OSS software package.
Sarovar.org provides this facility using Mailman.

\subsection{Surveys}

Sometimes feature implementations and even major turning points are
based on uses' preferences. Sarovar.org provides a tool to create
surveys and to cast votes.

\subsection{News Section}

There is a news section for each project at Sarovar.org in which the
project managers/developers can post news related to the hosted
project. Depends on the popularity of the project news items may be
qualified for Sarovar.org front page.

\subsection{Search Facility}

Sarovar.org provides a search facility. One can search projects,
people and skill set of registered users.

\subsection{Inviting Developers}

There is a section, for every hosted project, named ``Project
Openings.'' Project admins can use this facility if they need
additional help from other users/visitors of Sarovar.org.

\subsection{Code Snippets}

Developers often create smart, elegant yet small programs and scripts.
Because of the simplicity such a work cannot be qualified for a
project by itself. In such cases they can share it with the community
through the ``Code snippets'' section of Sarovar.org.

\subsection{General Information about Sarovar.org}

\subsubsection{Technical Information}
URL: http://sarovar.org\\
Platform: GNU/Linux Debian distribution.\\
Application Server: Apache, PHP and CGI\\
Hardware: Compaq PIV\\
ISP: Asianet Satcom Physical Location: Trivandrum, India

\subsubsection{Who We Are\label{1}}
Sponsor: River Valley Technologies (http://www.river-valley.com)\\
Technology: Linuxense Information Systems
Pvt. Ltd. (http://www.linuxense.com)\\


\hfill Prepared by Anil Kumar K. (anil@linuxense.com), Linuxense


\end{document}


